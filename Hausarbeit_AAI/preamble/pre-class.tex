\documentclass[10pt,a4paper]{article}
\usepackage{fontspec}
\defaultfontfeatures{Mapping=tex-text}
\usepackage{xunicode}
\usepackage{xltxtra}
%\setmainfont{???}
\usepackage{amsmath}
\usepackage{amsfonts}
\usepackage{amssymb}
\usepackage{graphicx}
\begin{document}
•
\end{document}% ------------------------------------------------------------------------
% LaTeX - Preambel ******************************************************
% ------------------------------------------------------------------------
% Dokumentklasse (Koma Script)
% ------------------------------------------------------------------------
% basiernd auf www.matthiaspospiech.de/latex/vorlagen Diplomarbeit kompakt
% ========================================================================
\documentclass[%
   %draft,            % Entwurfsstadium
   final,             % fertiges Dokument
   11pt,              % Schriftgroesse der Grundschrift
   bigheadings,       % große Überschriften
   ngerman,           % wird an andere Pakete weitergereicht
   a4paper,           % Papierformat
   BCOR=5mm,          % Bindekorrektur: Zusätzlicher Rand auf der Innenseite
   DIV=12,            % Seitengröße (siehe Koma Skript Dokumentation !)
   1.1headlines,     % Zeilenanzahl der Kopfzeilen
   pagesize,         % Schreibt die Papiergroesse in die Datei.
   oneside,          % Einseitiges Layout
%   twoside,          % Zweiseitiges Layout
   openright,        % Kapitel beginnen immer auf der rechten Seite
   titlepage,        % Titel als einzelne Seite ('titlepage' Umgebung)  
   headsepline,      % Linie unter Kolumnentitel ()
%   plainheadsepline, % Linie unter Kolumnentitel () plain Seitenstil
   nochapterprefix,  % keine Ausgabe von 'Kapitel:'
   bibtotoc,         % Bibliographie ins TOC
%	bibtotocnumbered, % Bibliographie ins TOC mit Kapitelnummer
   tocindent,        % eingereuckte Gliederung
   listsindent,      % eingereuckte LOT, LOF
   pointlessnumbers, % Überschriftnummerierung ohne Punkt, siehe DUDEN !
   cleardoubleempty, % Leere linke Seite bei Zweiseitenlayout vor Kapitel
   fleqn,            % Formeln werden linksbuendig angezeigt
%   parindent,        % Absatz mit Einzug (Standard)
   halfparskip,      % Absatz halbe Zeile Abstand
%   parskip,          % Absatz ganze Zeile Abstand
]{scrbook}%     Klassen: scrartcl, scrreprt, scrbook
