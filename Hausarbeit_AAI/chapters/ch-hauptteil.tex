% Bsp. eines Hauptteils
\newpage
\chapter{Theoretische Grundlagen}
\label{sec:grundl}

\section{Künstliche Intelligenz}
Künstliche Intelligenz ist ein Teilgebiet der Informatik und bezeichnet ein Computerprogramm oder System, welches mittels ein künstliches neuronales Netzwerk (KNN) menschliche Fähigkeiten imitiert. Wie beispielsweise das Logische Denken, Lernen, Kreativität, Mustererkennung. Dies ermöglicht solchen Systemen Probleme zu lösen, ohne dass menschliche Eingriffe erforderlich ist. Daher werden KI-Systeme in verschiedenen gebieten angewendet wie, z.B. in der Spracherkennung, in der Bilderkennung, im autonomen Fahren und in der medizinischen Diagnose und mehr \cite{doc:ki}. \\


\section{Generative Pre-trained Transformer (GPT)} \label{GPT}
Generative Pre-trained Transformer (GPT) ist eine Familie von state-of-the-art Tiefen Neuronalen Netzen, die speziell für die Verarbeitung natürlicher Sprache entwickelt wurden. Im Jahr 2018 veröffentlichte die ehemalige amerikanische Non-Profit-Organisation OpenAI die erste GPT Version und zuletzt die Version GPT-4 im März 2023 \cite{doc:DBLP,doc:GPT3,doc:GPT4}. \\ 

Die Skalierbarkeit\footnote{Flexible Anzahl der Parameter im Modell} der GPT-Netze  ermöglicht es sehr komplexe sprach Aufgaben zu bewältigen. In Kombination mit der Unsupervised Learning Fähigkeiten ist das Netz in der Lage, ohne aufwändige menschliche Datenvorverarbeitung zu Trainieren. Was dem Netz ermöglicht eine riesige Datenmenge als Trainingsdaten zu nutzen, wie z.B. Wikipedia Artikel, Zeitschriften, Webseiten, Bücher und mehr \cite{doc:DBLP}.  

Diese Fähigkeit des GPT-Netzes zur Nutzung großer Datenmenge als Trainingsdaten führte zu vielen bemerkenswerten Ergebnissen. Das Modell ist damit in der Lage menschliche Sprachproduktionen zu erzeugen, einschließlich lyrische Gedichte, Geschichten, Computerprogramme und Design. Darüber hinaus ist das Modell in der Lage sprachliche Aufgaben zu bearbeiten mit eine Menge an Fachwissen.

Zahlreiche Paper berichten von dem Potential dieser Architektur in Bereichen, wie Medizin, Recht, IT, Mathematik, Prozessirrung und mehr. \cite{doc:exp_paper}

Mit der Veröffentlichung von GPT-4 hat die Architektur neue Maßen angenommen, den zu der Verbesserten Leistung, ist die GPT-4-Architektur in der Lage Bilder und Audio zu verarbeiten und zu generieren.


\chapter{Rahmenbedingungen und Konzept}
\label{ch:RuK}
In diesem Kapitel sollen die Rahmenbedingungen, Herausforderungen und Beschränkungen im Zusammenhang mit der Integration von ChatGPT in den Entwicklungsprozess von RD-Abteilungen diskutiert werden. Dabei wird auch auf das Gesamtsystem eingegangen, in dem ChatGPT integriert ist, um die Funktionalität und Einsatzmöglichkeiten des Systems zu verdeutlichen.

\section{Konzept und Vorgehensweise}
Das Konzept dieser Arbeit besteht darin, durch gezielte Aufgabenstellungen an ChatGPT zu ermitteln, wozu die KI in der Lage ist und welche Aufgabenbereiche sie abdecken kann. Dazu wurden Aufgaben konstruiert, die in der Regel von RD-Mitarbeitern im Tagesgeschäft gelöst werden. Zudem wurde ein simulatives Konzept verwendet, um zu verdeutlichen, wie ChatGPT zur Verbesserung der internen Effizienz beitragen kann.

Aufgrund der Einschränkungen in \autoref{einschrnkung} durch die Richtlinien von Mercedes-Benz wurde das Experiment simulativ durchgeführt und die Ergebnisse auf Mercedes-Benz übertragen.

\section{Rahmenbedingungen} \label{einschrnkung}
Für die vorliegende Arbeit wurden verschiedene Versionen von ChatGPT getestet, darunter auch die kommerziell erhältliche GPT-4-Version. Die GPT-4-Version gilt als eine der fortschrittlichsten Versionen des GPT-Modells und bietet verbesserte Leistung und Funktionen im Vergleich zu ihren Vorgängerversionen. \\
Die Eingabe von Suchanfragen und Informationen in ChatGPT ist für OpenAI-Forscher und Experten sichtbar  \cite{doc:interneswissen}.
Dies steht jedoch im Widerspruch zu den Mercedes-Benz Richtlinien, die den Schutz von internes Wissen betonen. Daher wurde bewusst darauf verzichtet dies zu tun.


\chapter{Methoden und Durchführung von Experimenten}\label{sec:Methoden und Durchführung von Experimenten}
Die Auswirkungen von ChatGPT auf das Arbeitsleben bei Mercedes-Benz Forschung und Entwicklung können nur spekulativ für die Zukunft prognostiziert werden. Um eine genauere Einschätzung zu erhalten, war es jedoch wichtig, ChatGPT genauer zu untersuchen.

Um ChatGPTs Fähigkeiten besser einschätzen zu können, wurden ihm verschiedene Aufgaben aus übergeben, die typischerweise von Mitarbeitern im Rahmen ihrer täglichen Arbeit in Forschung und Entwicklung gelöst werden. Die Aufgaben wurden entsprechend ihrem Schwierigkeitsgrad, ihrer Komplexität, der erwarteten Erfahrung und der voraussichtlichen Dauer zur Lösung bewertet. Anschließend wurden sie über die Texteingabe der Buntzieroberfläche in  ChatGPT eingespeist und die Ergebnisse ausgewertet. \\

Folgende Aufgaben wurde ChatGPT gestellt:


\begin{table}[h]
\centering
\begin{tabular}{|c|c|c|c|}
\hline
\textbf{Aufgabentyp} & \textbf{Durchgeführt in Kapitel}  \\ \hline
Programmieraufgabe & \autoref{prog1}  \\ \hline
Sehr schwere Programmieraufgabe & \autoref{prog2} \\ \hline
Informationsbeschaffungstool & \autoref{informationsbeschaffungstool-t}\\ \hline
\end{tabular}
\caption{Experimententabelle}
\label{tab:beispiel}
\end{table}


\section{Programmier-Experiment: json-to-csv.py}
\label{prog1}
Die erste Programmieraufgabe bestand aus folgende aufgaben:
Ein internes Software-Anwendung generiert Json\footnote{Ein Datenaustauschformat das für Menschen verständlich ist} Dateien. Die Json Datein sind jedoch nicht einheitlich strukturiert und erfordern Datenbereinigung. Die Aufgabe von ChatGPT war es ein Python\footnote{Eine Programmiersprache} Programm zu schreiben, welches alle Datein auf deren Inhalt untersucht, den nötigen Daten extrahiert und diese strukturiert in einer Excel Tabelle zu speichern. Hierbei sollen leere Felder, mit den Wert NULL in der Tabelle auftauchen.


Der Schwierigkeitsgrad der Aufgabe wurde in \autoref{bew:json-to-csv} bewertet:
\begin{table}[h]
\centering
\begin{tabular}{|c|c|}
\hline
\textbf{Kategorie} & \textbf{Bewertung}  \\ \hline
Komplexität & Mittlere Komplexität \\ \hline
Erwartete Erfahrung & Gute Programmierkennrisse  \\ \hline
Voraussichtliche Dauer & Zwei Tage \\ \hline
\textbf{Gesamtbewertung} & Mittelschwere Aufgabe \\ \hline
\end{tabular}
\caption{Bewertung der json-to-csv Aufgabe}
\label{bew:json-to-csv}
\end{table}

\subsubsection{Versuchsdurchführung}
Diese Programmieraufgabe erfordert in der Regel einen Aufwand von 2 Tagen, neben anderen Tätigkeiten. Die Aufgabe wurde in mehreren Teilen an ChatGPT gestellt. Zuerst wurde die Input-Datei als Text für ChatGPT dargestellt. Danach wurde die Fragestellung und die eigentliche Aufgabe formuliert: \\

\emph{"Hallo ChatGPT! Könntest du mir ein Python-Programm schreiben, das die Informationen aus der eingegebenen JSON-Datei extrahiert, in einem Pandas\footnote{Eine Python Bibliothek}-Dataframe speichert und dann in einer CSV-Datei speichert? Wenn eine Variable nicht gefunden werden kann, soll sie durch den Wert NULL ersetzt werden."}

Daraufhin hat ChatGPT den Code generiert, der auf seine Richtigkeit überprüft wurde. Nachdem die Funktionalität des Codes bestätigt wurde, wurde ChatGPT darum gebeten, den Code um eine weitere Funktionalität zu erweitern.

\emph{"Es handelt sich nicht nur um eine JSON-Datei als Eingang, sondern viele, die sich alle in einem Ordner befinden. Könntest du den Code um diese Funktion erweitern?"}

Daraufhin hat ChatGPT den Code um die neue Funktion erweitert. Die Richtigkeit des Codes wurde überprüft und bestätigt. Der generierte Code befindet sich in \autoref{json-to-csv}.

\subsubsection{Versuchsauswertung}

Der Versuch hat in etwa 20 min gebraucht. Dazu gehört das eintippen der Aufgabe, Speichern und Ausführen des Codes, Testen und Bewertung der Leistung.

Dabei hat ChatGPT keine Rückfragen gestellt und den Code beim ersten Versuch geschrieben. Der Generierte Programmcode hat dazu die gestellte Aufgabe voll umfänglich erfüllt.



\section{Programmier-Experiment: Amazon.py}
\label{prog2}
Nachdem die Aufgabe im \autoref{prog1} von der künstlichen Intelligenz ohne Schwierigkeiten gelöst wurde, wurde ChatGPT eine äußerst anspruchsvolle Aufgabe gestellt. Die Herausforderung bestand darin, eine Webseite zu programmieren, die ähnliche Dienste wie Amazon\footnote{Börsennotierter Onlineversandhändler} anbietet.

Der Schwierigkeitsgrad der Aufgabe wurde in \autoref{bew:amazon} bewertet:

\begin{table}[h]
\centering
\begin{tabular}{|c|c|}
\hline
\textbf{Kategorie} & \textbf{Bewertung}  \\ \hline
Komplexität & Hochkomplex \\ \hline
Erwartete Erfahrung & Sehr gute Programmierkennrisse  \\ \hline
Voraussichtliche Dauer & 4-6 Monate \\ \hline
\textbf{Gesamtbewertung} & Sehr Schwere Aufgabe \\ \hline
\end{tabular}
\caption{Bewertung der Aufgabe Amazon.py}
\label{bew:amazon}
\end{table}

\subsubsection{Versuchsdurchführung}
Die hochkomplexe Programmieraufgabe wurde ChatGPT wie folgt vorgelegt: 
\emph{"Erstelle den Python-Code für eine Webseite, die ähnliche Dienste wie Amazon anbietet."} Jedoch hat ChatGPT nicht mit Code geantwortet, sondern mit einem Text: \emph{ "Als KI kann ich dir keine vollständige Python-Code-Lösung erstellen [...]".}

Daraufhin wurden Teilaspekte definiert, die zur Lösung dieser Aufgabe beitragen könnten, und ChatGPT wurden sie vorgelegt. In diesem Fall war ChatGPT in der Lage, mit Code zu antworten.

Allerdings wurde der Versuch aufgrund zeitlicher Einschränkungen und mangelnder Expertise nicht vollständig durchgeführt.

\subsubsection{Versuchsauswertung}
Die Versuchsdauer betrug etwa 15 Minuten. Aufgrund ihrer enormen Komplexität war es nicht möglich, ChatGPT mit der Erstellung einer hochkomplexen Software-Architektur zu beauftragen. Dies zeigt, dass KI-Systeme noch nicht in der Lage sind, solche Aufgaben automatisiert zu bewältigen.

Dennoch konnte ChatGPT bei der Lösung von Teilaufgaben hilfreich sein. Allerdings erforderte dies eine genaue technische Beschreibung der Aufgabenstellung und eine Expertise in Software- und Webentwicklung. ChatGPT war somit eine wertvolle Unterstützung und trug dazu bei, dass die Grundarchitektur einer Webseite innerhalb weniger Minuten erstellt werden konnte.


\section{ChatGPT als Informationsbeschaffungstool}
\label{informationsbeschaffungstool-t}
Wie bereits in \autoref{GPT} erwähnt wurde, ist ChatGPT in der Lage, Informationen sehr effektiv zusammenzufassen und den wichtigen Kern herauszufiltern.

\subsubsection{Einführung}

\begin{center}
\begin{minipage}{12cm}
\begin{quotation}
\textit{\enquote{Wenn Daimler wüsste, was Daimler weiß}}
\end{quotation}
\hfill \textsf ~ Ehemaliger Daimler CEO Dr. Dieter Zetsche
\end{minipage}
\end{center}

Wie der ehemalige CEO Dr. Zetsche betont, unterstreicht dies die Bedeutung eines reibungslosen Informationsflusses in einem Unternehmen. Ein Mangel an Informationen kann oft dazu führen, dass Arbeit unnötigerweise doppelt stattfinden, was die Effizienz des Unternehmens beeinträchtigt. Deshalb ist es eine wichtige Aufgabe von Forschungs- und Entwicklungsmitarbeitern, vorhandene Informationen zu kennen, zu suchen und anzuwenden, um solche Ineffizienzen zu vermeiden und den Erfolg des Unternehmens zu fördern.

Um sicherzustellen, dass keine Informationen verloren gehen und dass andere Mitarbeiter darauf Zugriff haben, werden relevantes Wissen unter anderem in Confluence dokumentiert. Da jeder Mitarbeiter für sich entscheidet was relevantes Wissen ist und in Confluence dokumentiert, sind ähnliche Informationen an vielen Stellen zu finden, was zu Redundanzen und einer übermäßigen Anzahl an Dokumentationen führt. Eine einzelne Person kann in ihrem Berufsleben nicht alle diese Dokumentationen lesen. Obwohl Confluence eine Suchfunktion bietet, kann es immer noch zeitaufwendig sein, aus den vielen Dokumentationen die wichtigen Informationen zu finden.

In diesem Experiment wurde untersucht, ob ChatGPT eine Lösung für dieses Problem darstellt, indem es wichtige Informationen extrahiert, Daten clusterisiert und auf komplexe interne Fragen anhand einer verfügbaren Datenbasis beantwortet.

\subsubsection{Versuchsdurchführung}
Um eine Referenz zu haben, wurde ChatGPT nach der Bedeutung interner Begriffe gefragt. In Antwort darauf erklärte ChatGPT, dass es zusätzlichen Kontext benötigt, um eine Antwort darauf zu liefern.

Seit \autoref{GPT} ist bekannt, dass ChatGPT in der Lage ist, Texte zusammenzufassen und wichtige Informationen zu extrahieren. In diesem Experiment soll ChatGPT anhand einer größeren und internen Datenbasis befragt werden. Da ChatGPT jedoch nur maximal 25.000 Zeichen als Eingabe akzeptieren kann, ist es nicht möglich, die gesamte Confluence Datenmenge in Textform an ChatGPT zu übergeben.

Um diese Herausforderung zu bewältigen, gibt es verschiedene Möglichkeiten \cite{doc:chatgptinternal}:

\begin{description}
\item [Fine-Tuning:] Das GPT-Modell kann anhand der internen Daten weiter trainiert werden, um seine Fähigkeit zur Verarbeitung und Kennung dieser Daten zu ermöglichen.
\item [In-Context-Learning:] ChatGPT kann mit zusätzlichem Kontext versorgt werden, um die internen Fragen anhand des Kontext beantworten zu können.
\end{description}

Fine-Tuning ist eine Möglichkeit, die potenziell zu besseren Ergebnissen führen kann. Allerdings ist es eine zeit- und kostenaufwändige Methode, die auch Expertenwissen erfordert \cite{doc:OpenAI-price}. Aus diesem Grund wurde diese Methode in diesem Fall nicht weiterverfolgt.

Im Vergleich dazu erfordert die Methode des In-Context-Learnings lediglich zusätzlichen Kontext. In der Praxis kann dies erreicht werden, indem die Anfrage nicht direkt an ChatGPT gestellt wird, sondern über Confluence in Form einer Suchanfrage formuliert wird. Die relevantesten Ergebnisse werden ausgewählt und ChatGPT zur Verfügung gestellt, um die Frage zu beantworten. Dieser Prozess kann automatisiert werden und würde in unserem Beispiel wie folgt aussehen: 

\begin{figure}[H]
\centering
\fbox{\includegraphics[width=0.9\textwidth]{fig/In-context_lerning.png}}
\caption{Beispielhafter Prozess zur Anwendung der In-Context-Learning Methode}
\label{fig:In-context_lerning}
\end{figure}

Um sicherzustellen, dass keine internen Informationen verwendet werden, wurde das Experiment in simulierter Form durchgeführt. Hierzu wurden mithilfe von ChatGPT falsche Informationen generiert (Siehe \autoref{Generation-NDAS}) und diese Informationen als Kontext für die gestellten Fragen verwendet:

Frage: \emph{"Kannst du anhand des folgenden Kontexts mir die Frage beantworten: Für was steht NDAS?
NDAS - Nachfolger der Assistenzsysteme [..]" }

Antwort: \emph{"NDAS steht für Nachfolger der Assistenzsysteme und beschreibt ein Konzept für ein neues, intelligentes Assistenzsystem in der Automobilbranche. Das System basiert auf künstlicher Intelligenz und bietet eine Vielzahl von Funktionen und Anwendungen [...]" }

\subsubsection{Versuchsauswertung}
 Dieses System hat das erwartete Ergebnis erbracht, indem ChatGPT die eingegebenen 300 Wörter als Wissensgrundlage verwendet hat, um auf die gestellte Frage einzugehen und die Antwort ausführlich zu erklären. Somit gilt dieser Versuch als erfolgreich bestanden.

\chapter{Evaluierung und Zusammenfassung}
\label{sec:evaluierung}
Die Evaluation wird sich darauf konzentrieren, wie sich der Einsatz von Künstlicher Intelligenz auf das Arbeitsleben der Mitarbeiter bei Mercedes-Benz Forschung und Entwicklung auswirkt. Hierbei dienen die Ergebnisse der durchgeführten Experimente in \autoref{sec:Methoden und Durchführung von Experimenten} sowie wissenschaftliche Erkenntnisse als Argumentationsgrundlage.

\section{Evaluation der Programmieraufgaben}

\subsubsection{Erkenntnisse den Programmierexperimenten}
Anhand des vorgeführten Programmierexperiments in \autoref{prog1} lässt sich folgendes feststellen: ChatGPT ist in der Lage, kleinere Programmieraufgaben robust und effektiv zu lösen, indem es vorhandene Code-Programme und -Beispiele aus dem Internet zusammensetzt und an die Anforderungen anpasst.

Dadurch können Entwickler im Arbeitsleben bei einfachen Aufgaben entlastet werden. Es ist jedoch wichtig zu betonen, dass Entwickler sich nicht vollständig auf ChatGPT verlassen werden und weiterhin programmieren müssen. ChatGPT übernimmt lediglich das Suchen nach passenden Codes und die teilweise Integration der Funktionen, während Entwickler zuvor ähnliche Aufgaben eigenständig durch Google-Suche und Verwendung von vordefinierten Bibliotheken gelöst haben.

Zusammenfassend bedeutet dies, dass Entwickler durch die Unterstützung von ChatGPT effizienter programmieren können, da sie sich weniger mit Syntax, Bibliotheken und der Suche nach Code-Beispielen befassen müssen. Dies ermöglicht Entwicklern, einen größeren Überblick über die Gesamtprozesse zu haben und mehr Verständnis für diese zu entwickeln. Dies wird voraussichtlich dazu führen, dass Entwickler kreativer und effektiver mit ihren Programmen umgehen können.

Die Fähigkeit, Informationen im Internet zu finden, ist heute für Entwickler von großer Bedeutung. Doch in Zukunft wird es noch wichtiger werden, dass Entwickler in der Lage sind, technische Prozesse und Funktionen in menschlicher Sprache zu beschreiben, damit sie von künstlicher Intelligenz wie ChatGPT verstanden werden können. Dies ist insbesondere relevant für die Code-Generierung, bei der Entwickler die Anforderungen an die Software in natürlicher Sprache formulieren.


\subsubsection{Die Grenzen von ChatGPT}
Die Ergebnisse des Experiments in \autoref{prog2} zeigen, dass ChatGPT nicht in der Lage ist, größere Projekte eigenständig zu bearbeiten. Dafür benötigt die künstliche Intelligenz die Anleitung eines menschlichen Mentors, der sie Schritt für Schritt durch die technischen Anforderungen führt.

Dies impliziert, dass größere Projekte auch weiterhin in menschlicher Verantwortung liegen werden. Die AI wird lediglich als zusätzliche Hilfe eingesetzt, um kleinere Programmieraufgaben zu erledigen.

\subsubsection{Vorteile durch KI für Informationsbeschaffung}
Wie in \label{informationsbeschaffungstool} gezeigt wurde, kann künstliche Intelligenz dazu beitragen, den Informationsfluss innerhalb eines Unternehmens zu verbessern. Dadurch ergeben sich verschiedene Vorteile, wie zum Beispiel:

\begin{itemize}
\item Erhöhung der Effizienz der Mitarbeiter, indem redundante Arbeiten, wie zum Beispiel die Entwicklung von Software-Tools, die Erstellung von Dokumentationen und Prozessen wegfallen.
\item Erhöhung der Effizienz durch die Reduktion der Suchzeit nach Informationen.
\item Reduktion des menschlichen Betreuungsbedarfs durch die Vereinfachung der Informationen durch ChatGPT.
\item Reduktion des Arbeitsaufwands und Verbesserung des Durchblicks durch des Unternehmens durch die automatisierten Zusammenfassungsfunktion von ChatGPT.
\end{itemize}


\subsubsection{Gesamt Evaluierung}
Basierend auf den bisherigen Annahmen ist zu erwarten, dass die Anzahl der Programmierer innerhalb eines Unternehmens, welche an eine bestimmte Aufgabe arbeiten, sich reduziert wird, da eine künstliche Intelligenz wie ChatGPT in der Lage ist, kleinere Programmieraufgaben effizient zu erledigen.
Allerdings werden Entwickler weiterhin für die Umsetzung größerer Projekte benötigt.

In ähnlicher Weise wird voraussichtlich auch die Auslagerung von Programmieraufgaben zurückgehen. Den In der Regel benötigen Dienstleister eine detaillierte technische Beschreibung, um eine Aufgabe zu erledigen. Eine künstliche Intelligenz wie ChatGPT kann solche Beschreibungen lesen und daraus Code generieren, was dazu führen könnte, dass Unternehmen weniger auf externe Dienstleister angewiesen sind und mehr mittels KI In-House-Code produzieren.

Ein verbesserter Informationsfluss würde dazu führen, dass weniger redundante Arbeiten anfallen. Dadurch würde die Effizienz der Mitarbeiter gesteigert werden, was wiederum zu einer Reduktion der benötigten Arbeitskräften führen könnte.
Zusammenfassend lässt sich sagen, dass die Verwendung von künstlicher Intelligenz für ein Unternehmen von großem Nutzen ist.

\subsection{Zusammenfassung}
In dieser Hausarbeit wurde untersucht, welche Auswirkungen die Verwendung von ChatGPT und ähnlicher künstlicher Intelligenz auf das Arbeitsleben von RD-Mitarbeitern hat. Hierfür wurden wissenschaftliche Dokumentationen herangezogen und verschiedene Experimente durchgeführt, deren Ergebnisse ausgewertet wurden.

Es wurde festgestellt, dass ChatGPT für Unternehmen sehr nützlich ist, da die Verwendung der künstlichen Intelligenz zu einer Steigerung der Effizienz und zu einer Senkung der Kosten führt. Allerdings wurde die These widerlegt, dass durch die Verwendung von ChatGPT keine Programmierer mehr benötigt werden. Stattdessen wird sich die Anzahl der Entwickler verringern.

Insgesamt zeigt diese Arbeit, dass die Verwendung von ChatGPT und ähnlicher künstlicher Intelligenz für Unternehmen ein großes Potenzial hat, um Prozesse zu optimieren und Kosten zu senken. Jedoch müssen auch die Einschränkungen und Grenzen der künstlichen Intelligenz beachtet werden, um eine sinnvolle Integration in den Arbeitsalltag zu gewährleisten.