\chapter{Anhang}
\label{sec:a-kapitel}
\section{Projekt und Vorstellung des Teams \glqq Data-Collection\grqq{} }



\section{Json to CSV Code}
Der folgende Code wurde von ChatGPT generiert. Er überprüft alle Dateien in einem Ordner auf ihre Eigenschaften, z.B. ob sie vom Typ \texttt{JSON} sind. In einer Schleife durchläuft er alle Dateien, extrahiert die Informationen und speichert sie in einer Excel-Tabelle. Der Code wurde auf Richtigkeit und Funktionalität geprüft.

\begin{lstlisting}[language=Python, caption=ChatGPT generierter Python-Code json-to-csv, label=json-to-csv]
import os
import pandas as pd
import json

# Ordnerpfad, in dem die JSON-Dateien gespeichert sind
folder_path = './json_files/'

# Liste aller JSON-Dateien im Ordner
json_files = [f for f in os.listdir(folder_path) if f.endswith('.json')]

# Liste der Dataframes fuer jede JSON-Datei
dfs = []

# Schleife ueber alle JSON-Dateien im Ordner
for file in json_files:
    file_path = os.path.join(folder_path, file)

    # Laden des JSON-Files
    with open(file_path, 'r') as f:
        data = json.load(f)

    # Extrahieren der Daten aus dem JSON-File
    timestamp = data.get('timestamp', 'NULL')
    adtf_config_version = data['ADTF'].get('Config-Version', data['ADTF'].get('Config1Version', 'NULL'))
    adtf_update_time = data['ADTF'].get('Update-Time', 'NULL')
    acroname_serial_nr = data['Acroname'].get('Serial-nr', 'NULL')
    acroname_sw_version = data['Acroname'].get('SW-Version', 'NULL')
    dspc_hostname = data['DSPC'].get('Hostname', 'NULL')
    dspc_car_id = data['DSPC'].get('Car ID', 'NULL')
    dspc_car_vin = data['DSPC'].get('Car VIN', 'NULL')
    dspc_saltenv = data['DSPC'].get('Saltenv', 'NULL')
    dspc_pillarenv = data['DSPC'].get('Pillarenv', 'NULL')
    gateway_sw_version = data.get('Gateway', {}).get('SW Version', 'NULL')
    nv_recorder_b_build_info = data.get('NV_Recorder_B', {}).get('Build Info', 'NULL')

    # Erstellen des Pandas-Dataframes
    df = pd.DataFrame({
        'timestamp': [timestamp],
        'ADTF_Config_Version': [adtf_config_version],
        'ADTF_Update_Time': [adtf_update_time],
        'Acroname_Serial_NR': [acroname_serial_nr],
        'Acroname_SW_Version': [acroname_sw_version],
        'DSPC_Hostname': [dspc_hostname],
        'DSPC_Car_ID': [dspc_car_id],
        'DSPC_Car_VIN': [dspc_car_vin],
        'DSPC_Saltenv': [dspc_saltenv],
        'DSPC_Pillarenv': [dspc_pillarenv],
        'Gateway_Sw_Version': [gateway_sw_version],
        'NV_Recorder_B_Build_Info': [nv_recorder_b_build_info]
    })

    dfs.append(df)

# Kombinieren aller Dataframes in eine einzige Datei
df_combined = pd.concat(dfs, ignore_index=True)

# Laden des CSV-Files (falls vorhanden)
try:
    df_existing = pd.read_csv('data.csv')
except FileNotFoundError:
    df_existing = pd.DataFrame()

# Kombinieren des bestehenden Dataframes und des neu erstellten Dataframes
df_combined = pd.concat([df_existing, df_combined], ignore_index=True)

# Speichern des kombinierten Dataframes in eine CSV-Datei
df_combined.to_csv('data.csv', index=False)

\end{lstlisting}



\section{Versuch Amazon Webseite Code}

\begin{lstlisting}[language=HTML, caption=ChatGPT generierter HTML-Code amazon, label=Versuch-Amazon-Webseite]
<!DOCTYPE html>
<html>
<head>
	<title>My Website</title>
	<meta charset="UTF-8">
	<meta name="viewport" content="width=device-width, initial-scale=1.0">
	<link rel="stylesheet" href="style.css">
</head>
<body>
	<header>
		<h1>Welcome to my website</h1>
		<nav>
			<ul>
				<li><a href="#section-1">About me</a></li>
				<li><a href="#section-2">My projects</a></li>
				<li><a href="#section-3">Contact</a></li>
			</ul>
		</nav>
	</header>

	<main>
		<section id="section-1">
			<h2>About me</h2>
			<p>I am a passionate web developer who loves creating engaging websites. My focus is on developing responsive websites that look great on all devices and provide an excellent user experience.</p>
			<p>In my free time, I enjoy experimenting with new technologies and keeping up with the latest developments in web development.</p>
			<p>Currently, I work as a web developer at XYZ Webdesign and have worked on many exciting projects over the years. If you would like to learn more about my experience and skills, take a look at my resume.</p>
			<a href="resume.pdf">Download resume</a>
		</section>

		<section id="section-2">
			<h2>My projects</h2>
			<ul>
				<li>
					<h3>Project 1</h3>
					<p>I worked on a project to develop an e-commerce website for a client. The website was tailored to the client's needs and included features such as a product database, shopping cart, and order management.</p>
				</li>
				<li>
					<h3>Project 2</h3>
					<p>I worked on a project to develop an event website for a local organization. The website included information about upcoming events, a way to register online, and a discussion forum for participants.</p>
				</li>
				<li>
					<h3>Project 3</h3>
					<p>I worked on a project to redesign the website of a restaurant. The website included information about the menu, opening hours, and a way to make reservations online.</p>
				</li>
			</ul>
		</section>

		<section id="section-3">
			<h2>Contact</h2>
			<p>You can reach me by email at <a href="mailto:info@mywebsite.com">info@mywebsite.com</a> or by filling out the form below:</p>
			<form>
				<label for="name">Name:</label>
				<input type="text" id="name" name="name"><br>

				<label for="email">Email:</label>
				<input type="email" id="email" name="email"><br>

				<label for="message">Message:</label>
				<textarea id="message" name="message"></textarea><br>

				<input type="submit" value="Submit">
			</form>
		</section>
	</main>

	<footer>
		<p>&copy; 2023 My Website</p>
	</footer>

</body>
</html>



\end{lstlisting}

\subsection{Befragung ChatGPT zu internes Wissen}
\label{interne-fragen}


Frage: Für was steht "ndas" im Zusammenhang mit Mercedes-Benz


ChatGPT: Ich konnte keine spezifische Bedeutung von "ndas" im Zusammenhang mit Mercedes-Benz finden. Es ist möglich, dass es sich um eine interne Abkürzung oder einen Slang innerhalb des Unternehmens handelt, der nicht öffentlich bekannt ist. Ohne weiteren Kontext oder Informationen kann ich Ihnen leider keine genauere Antwort geben.

\subsection{Generation von falschen Informationen als Kontext für eine Anfrage}
\label{Generation-NDAS}


Frage: NDAS Steht für Nachfolger der Assistenzsysteme. Kannst du mir ein Artikel von 300 Wörter darüber schreiben.

ChatGPT:Gerne! Hier ist ein Artikel über NDAS:

NDAS - Nachfolger der Assistenzsysteme

In der Automobilbranche hat sich in den letzten Jahren viel verändert. Insbesondere die Entwicklung von Assistenzsystemen hat dazu beigetragen, dass Autos heute sicherer und komfortabler als je zuvor sind. Doch die Entwicklung steht nicht still, und es gibt bereits den nächsten Schritt in der Evolution der Assistenzsysteme: NDAS.

NDAS steht für Nachfolger der Assistenzsysteme und beschreibt ein Konzept für ein neues, intelligentes Assistenzsystem, das auf künstlicher Intelligenz basiert und eine Vielzahl von Funktionen und Anwendungen bietet. NDAS soll in Zukunft nicht nur die Sicherheit und den Komfort im Straßenverkehr erhöhen, sondern auch den Fahrspaß steigern und eine nahtlose Integration von Smart Home und Mobilität ermöglichen.

Das Herzstück von NDAS ist eine leistungsstarke, lernfähige künstliche Intelligenz, die in der Lage ist, die Umgebung des Fahrzeugs zu erkennen, zu interpretieren und darauf zu reagieren. Dabei greift NDAS auf eine Vielzahl von Sensoren und Kameras im Fahrzeug zurück, um ein 360-Grad-Bild der Umgebung zu erzeugen. Auf Basis dieser Daten kann NDAS den Fahrer in vielen Situationen unterstützen und ihm sogar Aufgaben abnehmen.

Zu den möglichen Funktionen von NDAS gehören unter anderem eine automatische Spurhaltung, eine adaptive Geschwindigkeitsregelung, eine automatische Einparkhilfe und eine Stauassistenz. Auch die Integration von Sprachsteuerung, Gesichtserkennung und Augmented Reality sind denkbar.

Ein weiterer Vorteil von NDAS ist die nahtlose Integration von Smart Home und Mobilität. Durch die Verknüpfung mit intelligenten Geräten im Haus und der Nutzung von Daten aus der Cloud kann NDAS beispielsweise das Licht und die Heizung im Haus steuern, bevor der Fahrer zuhause ankommt. Auch das Einkaufen oder das Bestellen von Lebensmitteln kann über NDAS erfolgen, während der Fahrer unterwegs ist.

Die Entwicklung von NDAS ist ein ehrgeiziges Projekt, das die Automobilbranche in den nächsten Jahren revolutionieren könnte. Die Integration von künstlicher Intelligenz, Sensoren und Smart-Home-Funktionen eröffnet viele neue Möglichkeiten und könnte den Straßenverkehr sicherer, komfortabler und effizienter machen. Die Zukunft des Fahrens sieht vielversprechend aus, und NDAS könnte ein wichtiger Schritt in Richtung autonomes Fahren und vernetzte Mobilität sein.