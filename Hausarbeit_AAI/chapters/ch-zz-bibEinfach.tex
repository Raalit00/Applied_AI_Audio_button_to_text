\begin{thebibliography}{XXXX}

\bibitem{doc:exp_paper}
Yiheng Liu, Tianle Han, Siyuan Ma, Jiayue Zhang, Yuanyuan Yang, Jiaming Tian, Hao He, Antong Li, Mengshen He, Zhengliang Liu, Zihao Wu, Dajiang Zhu, Xiang Li, Ning Qiang, Dingang Shen, Tianming Liu, and Bao Ge.
\newblock Summary of ChatGPT/GPT-4 Research and Perspective Towards the Future of Large Language Models.
\newblock {\em arXiv preprint arXiv:2304.01852}, 2023.

\bibitem{doc:GPT3} Tina Nord , Was ist GPT-3 und spricht das Modell Deutsch?,
\url{https://www.lernen-wie-maschinen.ai/ki-pedia/was-ist-gpt-3-und-spricht-das-modell-deutsch/}; \textit{zugriff am 26.04.2023} \textit{updated am 14. März 2023}

\bibitem{doc:GPT4} Tina Nord , Was du über GPT-4 wissen solltest,
\url{https://www.lernen-wie-maschinen.ai/sprachsuche/was-wir-bisher-ueber-gpt-4-wissen/}; \textit{zugriff am 26.04.2023} \textit{erstellt am  30. März 2023}


\bibitem{doc:OpenAI-ChatGPT} OpenAI , Pricing,
\url{https://openai.com/blog/chatgpt}; \textit{zugriff am 29.04.2023} \textit{erstellt am  November 30, 2022}

\bibitem{doc:OpenAI-price} OpenAI , Introducing ChatGPT,
\url{https://openai.com/pricinglanguage-models}; \textit{zugriff am 08.05.2023} 

\bibitem{doc:DBLP}author  = {Tom B. Brown and
                  Benjamin Mann and
                  Nick Ryder and
                  Melanie Subbiah and
                  Jared Kaplan and
                  Prafulla Dhariwal and
                  Arvind Neelakantan and
                  Pranav Shyam and
                  Girish Sastry and
                  Amanda Askell and
                  Sandhini Agarwal and
                  Ariel Herbert{-}Voss and
                  Gretchen Krueger and
                  Tom Henighan and
                  Rewon Child and
                  Aditya Ramesh and
                  Daniel M. Ziegler and
                  Jeffrey Wu and
                  Clemens Winter and
                  Christopher Hesse and
                  Mark Chen and
                  Eric Sigler and
                  Mateusz Litwin and
                  Scott Gray and
                  Benjamin Chess and
                  Jack Clark and
                  Christopher Berner and
                  Sam McCandlish and
                  Alec Radford and
                  Ilya Sutskever and
                  Dario Amodei}, Language Models are Few-Shot Learners,
\url{https://dblp.org/rec/journals/corr/abs-2005-14165.bib}; \textit{zugriff am Wed, 03 Jun 2020 11:36:54 +0200} \textit{erstellt am 29-03-2021}


\bibitem{doc:MBRD} Mercedes-Benz AG, Forschung \& Entwicklung.,
\url{https://group.mercedes-benz.com/karriere/ueber-uns/einblicke/research-development/}; \textit{zugriff am 30.04.2023} 

\bibitem{doc:100m} Deutschlandfunk, KI-Software ChatGPT ist am schnellsten wachsende Verbraucher-App der Geschichte,
\url{https://web.archive.org/web/20230203023335/https://www.deutschlandfunk.de/ki-software-chatgpt-ist-am-schnellsten-wachsende-verbraucher-app-der-geschichte-104.html}; \textit{zugriff am 27.04.2023} \textit{erstellt am 03-02-2023}

\bibitem{doc:confluence} Scolution ,Confluence – Funktionen und Einsatzmöglichkeiten,
\url{https://scolution.de/atlassian/confluence/};
\textit{zugriff am 30.04.2023}


\bibitem{doc:interneswissen} Natalie (OpenAI) ,What is ChatGPT? – Funktionen und Einsatzmöglichkeiten,
\url{https://help.openai.com/en/articles/6783457-what-is-chatgpt};
\textit{zugriff am 30.04.2023} \textit{Updated am 20.04.2023} 


\bibitem{doc:ki} Europäisches Parlament, Was ist künstliche Intelligenz und wie wird sie genutzt?,
\url{https://www.europarl.europa.eu/news/de/headlines/society/20200827STO85804/was-ist-kunstliche-intelligenz-und-wie-wird-sie-genutzt}; \textit{zugriff am 26.04.2023} \textit{erstellt am 29-03-2021}

\bibitem{doc:chatgptinternal} Quy Tang, Integrating ChatGPT with internal knowledge base and question answer platform Bring the power of ChatGPT to internal knowledge management,
\url{https://medium.com/singapore-gds/integrating-chatgpt-with-internal-knowledge-base-and-question-answer-platform-36a3283d6334}; 
\textit{zugriff am 05.05.2023} \textit{erstellt am 29-03-2023}


\end{thebibliography}
