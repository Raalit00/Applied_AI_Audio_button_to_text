\chapter{Einleitung}
\label{sec:einl}







\section{Motivation}
In einer Welt, in der Technologie und digitale Kommunikation immer mehr an Bedeutung gewinnen, wird die Möglichkeit, Tastenanschläge allein durch ihr Klangprofil zu identifizieren, als eine faszinierende technologische Errungenschaft betrachtet. Diese Fähigkeit birgt jedoch nicht nur innovative Anwendungsmöglichkeiten, sondern auch ernsthafte Sicherheitsrisiken. Insbesondere könnte eine Technologie zur akustischen Erkennung von Tastaturgeräuschen für Personen, die ihre Arbeit und Interaktionen häufig in digitalen Live-Streams oder in öffentlichen Räumen teilen, eine signifikante Bedrohung darstellen. Die potenzielle Gefahr, dass sensible Informationen – sei es durch Passwörter, private Nachrichten oder vertrauliche Daten – durch einfaches Abhören von Tastenanschlägen kompromittiert werden könnten, erfordert besondere Aufmerksamkeit.

Die subtilen Unterschiede in den Klangprofilen verschiedener Tastaturen oder spezifischer Tastenanschläge bleiben für das menschliche Ohr oftmals ununterscheidbar, doch der Fortschritt in der künstlichen Intelligenz und in spezialisierten Algorithmen zur Klassifizierung ermöglicht die Realisierung einer solchen Technologie. Die Kombination aus leistungsfähigen, tiefen Lernmodellen und fortschrittlichen Techniken in Computer Vision und -audition eröffnet neue Wege für die präzise akustische Analyse und Interpretation.

Das primäre Ziel dieser Arbeit liegt darin, die Realisierbarkeit und das damit verbundene Risikopotenzial dieser Technologie zu untersuchen. Durch die Entwicklung und das Training eines Convolutional Neural Networks (CNN) als Proof of Concept wird die Fähigkeit einer KI demonstriert, Tastenanschläge allein auf Grundlage ihres Klangs zu identifizieren. .




\section{Zielsetzung}
Das primäre Ziel dieser Arbeit liegt darin, die Realisierbarkeit und das damit verbundene Risikopotenzial dieser Technologie zu untersuchen. Durch die Erstellung eines Datensatzes, Entwicklung und das Training eines Convolutional Neural Networks (CNN) als Proof of Concept wird die Fähigkeit einer KI demonstriert, Tastenanschläge allein auf Grundlage ihres Klangs zu identifizieren. 

\section{Struktur dieser Arbeit}
Die vorliegende Arbeit ist in sechs Kapitel unterteilt. Das erste Kapitel soll eine Einführung in das Themengebiet liefern und als Einleitung fungieren. Nachfolgend werden in \autoref{sec:grundl} die Grundlagen erörtert und die Rahmenbedingungen aufgezeigt. Anschließend wird in \autoref{ch:RuK} das Konzept und die Arbeitsmittel vorgestellt. Im \autoref{sec:Methoden und Durchführung von Experimenten} werden die verwendeten Methoden zur Untersuchung von ChatGPT vorgestellt und schließlich in \autoref{sec:evaluierung} die Ergebnisse evaluiert und Prognosen getroffen. Zuletzt wird in \autoref{sec:Zus} der Inhalt dieser Arbeit zusammengefasst.

